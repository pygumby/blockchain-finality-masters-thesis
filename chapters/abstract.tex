In the wake of the discussion on central bank digital currencies (CBDC), the applicability of blockchain technology in the realm of payments is subject of debate.
However, an assessment of whether and how settlement finality can be achieved in blockchain systems is largely absent from the discourse, despite the direct impact of finality on settlement risks.

This thesis addresses this question.
First, a definition of finality based on a literature review is provided.
Then, as finality is traditionally regarded a legal concept, it is assessed whether blockchain-based payment systems are compatible with said notion of legal finality.
Lastly, as the legal account of finality implicitly assumes certain technical properties, blockchain technology is analyzed with regards to this notion of technical finality.

It is concluded that Bitcoin-style blockchain technology is both incompatible with legislation of settlement finality and unable to achieve technical finality.
However, changes in legislation on final settlement and the adoption of alternative blockchain technologies may enable blockchain-based payment systems to settle with finality.

% Im Zuge der Diskussion über digitale Zentralbankwährungen (CBDC) wird die Verwendbarkeit von Blockchaintechnologie im Bereich des Zahlungsverkehrs diskutiert.
% Ob und wie Settlementfinalität in Blockchainsystemen erreicht werden kann, bleibt im Diskurs jedoch weitgehend unbehandelt, trotz der direkten Auswirkung von Finalität auf Settlementrisiken.

% Die vorliegende Thesis befasst sich mit dieser Frage.
% Zunächst wird eine Definition von Finalität basierend auf einer Literaturübersicht gegeben.
% Da Finalität traditionell als rechtliches Konzept gilt, wird anschließend untersucht, ob blockchainbasierte Zahlungssysteme mit dieser rechtlichen Auffassung von Finalität vereinbar sind.
% Da der rechtliche Begriff von Finalität implizit bestimmte technische Eigenschaften voraussetzt, wird Blockchaintechnologie schliesslich im Hinblick auf diesen technischen Begriff der Finalität analysiert.

% Es wird geschlussfolgert, dass Blockchaintechnologie im Stile von Bitcoin sowohl mit dem rechtlichen Konzept der Finalität unvereinbar ist als auch nicht in der Lage ist, technische Finalität zu erreichen.
% Änderungen in der Gesetzgebung zur Finalität und die Anwendung alternativer Blockchaintechnologien können jedoch blockchainbasierte Zahlungssysteme, die mit Finalität settlen, ermöglichen.

\vspace{1cm}

\begin{tabularx}{\textwidth}{@{}l X}
  Keywords: & Payment systems, settlement systems, settlement finality, settlement risks, blockchain technology, distributed consensus, Bitcoin, proof of work, majority attacks, selfish mining, Byzantine agreement, Stellar consensus protocol
\end{tabularx}
