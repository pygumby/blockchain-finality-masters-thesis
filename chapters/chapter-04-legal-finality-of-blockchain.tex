As finality is traditionally seen as a legal concept, this chapter examines whether blockchain-based payment systems can provide said notion of legal finality.
To this end, a general overview of regulatory approaches to blockchain technology is given.
Then, it is assessed whether systems are covered by the SFD, which is used as an example of similar such laws once again.

\section{Regulatory approaches}

In the following, two approaches to regulating blockchain technology are characterized, namely what I consider ``indirect'' and ``direct regulation.''

\subsection{Indirect regulation}

At the time of this writing, blockchain systems have existed for well over a decade and, if the current Bitcoin bull run \autocite{nytimes2021} is any indication, are as popular as ever -- all the more reason for regulators to address the technology if they have not already.
In practice, regulators must choose between ``three paths [...] regarding regulation in this area,'' i.e., banning and/or restricting, ignoring or attempting to ``provide clarity and a regulatory framework regarding how such activities can operate within the jurisdiction.''  \autocite[210]{ellul2020}

While the first option is, due to the decentralized nature of blockchain systems, ``extremely hard, if not impossible, to enforce'' and the second option may lead to the emergence of ``legal uncertainty,'' driving away ``stakeholders,'' it is the third option -- ``legal certainty [...] provided through a regulatory regime'' -- that fosters innovation. \autocite[210]{ellul2020}
Note that albeit this assessment was originally made in the specific context of cryptocurrencies, it can, arguably, be applied to blockchain technology as a whole.

As to how regulators can and do go about providing said legal certainty with regards to blockchain technology, it is to be pointed out that ``[t]raditionally, the technology which enables a regulated financial product or service is considered to be outside the purview of the law -- it is the actions of the parties involved in providing or using the services that are to be regulated.'' \autocite[211]{ellul2020}
Consequently, it is possible for legislators to introduce legal certainty to the blockchain domain without having to explicitly address the blockchain technology that underpins the systems in question -- a practice that, in the remainder of this thesis, is referred to as ``indirect regulation of blockchain technology,'' for lack of better terminology.

This appears to be the go-to approach in many jurisdictions.
One example of this has already been presented, namely the Liechtenstein Blockchain Act, which does not directly target blockchain technology but defines roles for those that interact with systems based on it, e.g., the aforementioned physical validator role, and regulates those who conform to these roles.
Another example of this is what became known as the BitLicense, which any ``person (whether an individual or a company) that engages in Virtual Currency Business Activity'' in New York State, US, must apply for. \autocite{bitlicense2020}

The role that seems to be under most regulatory scrutiny is that of the exchanges where blockchain tokens and legal tender are traded, as ``for regulators Bitcoin exchanges are the most logical institutional choke point in the Bitcoin ecosystem.'' \autocite[1153]{tsukerman2015}
One of the main reasons for this is that the participants of many blockchain systems remain pseudonymous \autocite[6]{zhang2019} or anonymous \autocite{saberhagen2013}, which poses challenges, e.g., in the context of anti money laundering (AML).
Thus, exchanges -- which, by contrast, are legally tangible institutions -- are commonly required to identify their customers.
Generally speaking, exchanges are targeted by regulatory efforts because, in many respects, they are seen as the link between the oftentimes (pseudo-)anonymous on-chain world and traditional, well regulated financial infrastructures.
One example of this is the Foreign Accounts Tax Compliance Act (FATCA) in the US, as exchanges correspond to foreign financial institutions (FFIs) as defined therein, which ``are required to identify their US account holders to the Internal Revenue Service (IRS), or face a 30 percent gross tax on payments received from US sources.'' \autocite[1151]{tsukerman2015} 

It can be summarized that in order to not stifle but foster innovation, regulators have no choice but to provide legal certainty in the blockchain domain.
It appears as if the go-to approach for doing so is what can be referred to as indirect regulation of blockchain technology, i.e., the practice of regulating those who interact with blockchain  systems in a given jurisdiction instead of the technology that underpins such systems.
One of many examples of this is the Lichtenstein Blockchain Act.

\subsection{Direct regulation}

As emphasized previously, it is the facilitation of consensus in a decentralized setting that is the main contribution of blockchain technology.
However, ``the very same enabling features that bring decentrali[z]ation also pose challenges'' in the context of regulation. \autocite[209]{ellul2020}
While these challenges are circumvented by the aforementioned approach of indirect regulation -- where the technology itself, and with it its decentralized nature, are outside the purview of the law -- they become evident when examining the extent to ``which [...] government action can influence the blockchain payment systems in the first place,'' as ``the question whether state regulation can, if desired so, effectively restrain blockchain payment systems is a valid and a non-trivial topic to investigate.'' \autocite[2]{shanaev2019}
Such government-initiated influence or restraint directed at blockchain-based systems directly is, in the remainder of this thesis, referred to as ``direct regulation of blockchain technology'' --  again, for lack of better terminology.

One of the central of regulatory challenges arising from the decentralized structure of blockchain systems is the fact that such systems do ``not exist in a central location but rather through a peer-to-peer [...] network composed of all [...] users.'' \autocite[1128]{tsukerman2015}
To complicate matters further, the networks that make up such systems do not only span virtually all jurisdictions, they do so in a fluctuating way, as nodes enter and leave the system continuously.
Consequently, it can be said that blockchain systems transcend jurisdictions in the traditional, i.e., territorial, sense of the word and ``are necessarily exterritorial.'' \autocite[2]{shanaev2019}
They, thus, fall outside any national legal framework or regulatory regime.

Another central regulatory challenge is a direct consequence of decentralization as well:
``No particular party can be said to `control' the blockchain.'' \autocite[1129]{tsukerman2015}
In other words, there exists no one institution responsible for the operation of a blockchain system, as, again, said operation is done collectively by a transnational and fluctuating set of participants.
This, after all, was the main motivation for creating the original blockchain system, Bitcoin, in the first place, as it ``was launched on an ethos of anti-institutionalism.'' \autocite[6]{yeoh2017}
Consequently, it can be said that no legally liable operator exists for blockchain systems.

The absence of a tangible physical location as well as the absence of a legally liable operator are only two examples of how the decentralized structure of blockchain systems de-facto shields such systems from being impacted by means of law enforcement -- which, in turn, makes imposing legislation on them near pointless.
One could, thus, argue that direct regulation of blockchain-based systems, as opposed to regulating those who interact with them in a given jurisdiction, seems infeasible, at least as it stands today.

One of many examples of how blockchain technology is de-facto beyond the reach of law enforcement concerns data protection and privacy legislation as ``the immutability of [blockchain] records [...] could be in contradiction with'' Article 17 of the EU's General Data Protection Regulation (GDPR), i.e., the right to be forgotten \autocite[35]{esma2017}.
Yet, there appears to be no way to enforce the GDPR and actually have those blockchain records removed that are in breach of it.
Generally speaking, ``[t]he structure of blockchain records could [...] generate legal issues where regulators or laws could demand that erroneous or illegal transactions be unwound,'' \autocite[7]{yeoh2017} while the reality seems to be that such demands are, in fact, unenforceable.

It must be pointed out, however, that these remarks assume regulatory efforts to be confined to individual or groups of nation states and are not to deny that ``regulation, at least theoretically, can have a significant impact on [...] blockchain payment systems'' or, rather, blockchain systems in general.
One could, in fact, argue that ``regulation can be effective, especially if it is enforced internationally.'' \autocite[2]{shanaev2019}
As of today, however, this is not the case. 

It can be summarized that the decentralized nature of blockchain systems is, among other things, the reason why such systems fall outside any legal framework and lack legally liable operators, which, in turn, makes them impossible to come by through means of law enforcement and, therefore, de-facto unregulatable, at least in the sense of direct regulation.
This assessment is, however, predicated on the absence of internationally enforced regulation, as is the case today.

As a side note, for lack of a better place to do so, it must be pointed out that these anarchic characteristics did not come about accidentally.
The Bitcoin paper can only be understood as to propose a way to deliberately avoid ``mediating disputes'' -- legal disputes, one must assume -- as traditional ``financial institutions'' are unable to do so. \autocite[1]{nakamoto2008}
Among other factors, it is this open disdain for government regulation why David Golumbia, professor of humanities at the Virginia Commonwealth University, attributes Bitcoin and similar projects to the realm of ``cyberlibertarianism'' that is consistent with ``a holistic worldview that has been deliberately developed and promulgated by right-wing ideologues'': \autocite[5-6]{golumbia2016}

\begin{quote}
  By far the majority of interest in Bitcoin came from technologists [...].
  To those of us who were watching Bitcoin with an eye toward politics and economics, though, something far more striking than Bitcoin’s explosive rise in value became apparent:
  in the name of this new technology, extremist ideas were gaining far more traction than they previously had outside of the extremist literature to which they had largely been confined. [...]

  To anyone aware of the history of right-wing thought in the United States and Europe, [these ideas] are shockingly familiar:
  that central banking [...] is a deliberate plot to `steal value' from the people to whom it actually belongs; that the world monetary system is on the verge of imminent collapse due to central banking policies [...]; that `hard' currencies such as gold provide meaningful protection against that purported collapse; that inflation is a plot to steal money from the masses and hand it over to a shadowy cabal of `elites' who operate behind the scenes; and more generally that the governmental and corporate leaders and wealthy individuals we all know are `controlled' by those same `elites.'
\end{quote}

On a personal note, I must emphasize that I am appalled that blockchain technology is likely rooted in extreme right-wing ideology.
Although I did notice some clues along the way, e.g., the aforementioned sentiment of the Bitcoin paper and the support Nick Szabo -- someone in Satoshi Nakamoto's inner circle, after all -- for Donald Trump and his extreme right-wing conspiracy theories \autocite{szabo2020}, I was blinded by the elegance of the technology and, regrettably, needed to stumble upon the research quoted above to make sense of its political origins.

While I believe that blockchain technology has been and will continue to be used for good, I, too, believe that everyone dealing with it needs to be made aware of its likely origins.
In this thesis, this appears to be the most fitting place to do so.

\section{Legal finality in blockchain systems}

Whether a payment system is able to provide legal finality largely depends on the regulatory framework it is embedded in -- although, as mentioned before, technical aspects such as the interval of settlement cycles do, to some degree, have in impact.
Thus, in assessing whether blockchain-based payment systems can achieve final settlement, it is crucial to establish whether they fall under the protection of legislation aiming to enable payment systems to do so, e.g., the SFD, which is analyzed in the following.

Previously, it was pointed out that blockchain technology is commonly regulated indirectly, i.e., it is those who interact with it that are regulated rather than the technology itself, as is traditionally the case with technology underlying a financial product or service.
Furthermore, it was pointed out that regulating a blockchain system directly is likely infeasible, as its decentralized nature seems to place it beyond reach of law enforcement.
Thus, the question arises whether the SFD regulates payment systems in a indirect or direct fashion, since the latter case would likely render blockchain technology incompatible with it and, in turn, unable to achieve legal finality.

At a first glimpse, one could get the impression that the SFD continues the aforementioned tradition in that it does not impose regulation on a payment system directly, since its Articles enable a system's ability to achieve final settlement predominantly by limiting the effects of other legislation, e.g., insolvency law.

A closer look, however, reveals that the SFD is actually not technology-agnostic and, thus, does address payment systems directly, as it defines criteria and requirements for such systems that cannot be considered in isolation from their underlying technology -- at least not when it comes to blockchain-based payment systems.
In fact, the SFD only applies to a payment system in the first place if that system meets the Directive's definition of such \autocite[Article 2(a)]{eu1998} -- which may well have been conceived without technology in mind but is, nonetheless, very relevant as to whether or not a payment system based on blockchain technology qualifies.

% As one would expect, payment systems based on blockchain technology do not meet the SFD's definition of such, mainly because they are not govered by a Member State.
It is this very definition based on which one could argue that a blockchain-based system cannot constitute a payment system according to SFD, since the Directive defines such a system as a ``formal arrangement'' between ``participants'' that is ``governed by the law of a Member State'' and is ``designated [...] as a system and notified to the Commission by the Member State whose law is applicable.'' \autocite{eu1998}
However, a blockchain system such as Bitcoin is an informal arrangement between a fluctuating set of participants whose decentralized nature precludes it from being governed by the law of a Member State -- or the law of any other territorial jurisdiction, for that matter.
From this follows that there is no Member State to designate such a system and notify said designation to the Commission.
It can, thus, be said that the SFD's payment system definition excludes blockchain systems.

As stated before, for a blockchain-based payment system to not constitute such a system pursuant to the SFD means that the transactions it processes are not protected by the Directive, e.g., they are not shielded from the effects of applicable insolvency law(s) and are, effectively, opened up to the potentially disruptive impact of the latter.
Consequently, a blockchain-based payment system's transactions are not guaranteed to achieve finality in a legal sense.

As a side note:
If, in fact, the aforementioned laws, e.g., insolvency law, do apply to a blockchain-based payment system, then its decentralized structure inhibits their enforcement. 
Could one infer from this that a blockchain-based payment system is, by default, in conflict with certain laws?

% Permissioned blockchain technology may very well underpin a payment system pursuant to the SFD but is not considered further, as it essentially conforms to traditional, centralized technology.
These and similar considerations are likely the reason why some central banks have concluded that if blockchain technology ``were ever to form the core of large-value payment systems, that technology would be deployed in a permissioned manner.'' \autocite{liao2017}
Indeed---since permissioned blockchain technology is, by definition, centralized, it does not pose any of the previously identified hurdles to regulation and can, therefore, very well underpin a payment system that is designated as such pursuant to the SFD.
At this point, it is, however, worth repeating that the contribution of permissioned blockchain technology over traditional approaches to networking and databases is not clear, which is why the technology is not considered in this thesis.

However, one could dismiss the previously presented reasons as to why blockchain systems do not conform to the SFD as mere formal arguments.
After all, the SFD is more than two decades old and, accordingly, was not conceived with blockchain technology in mind.
What if the SFD's payment system definition were to be revised to encompass blockchain-based systems?
And, more generally speaking, what if the main hurdles to regulating blockchain systems were to no longer exist, e.g., because a way to internationally enforce laws were to emerge?

The central question that this line of thought boils down to is:
Given blockchain-friendly finality legislation and/or international law enforcement, would blockchain-based payment systems be able to achieve legal finality or would the application of the SFD and similar legislation still fail -- due to intrinsic properties of blockchain technology?

In this thesis, I argue that blockchain systems in the style of Bitcoin, i.e., PoW-based blockchain systems, would, in fact, still be incompatible with the SFD, because the concept of finality that the Directive implicitly assumes is in conflict with intrinsic properties of blockchain technology.

As previously established, ``settlement finality is generally defined in reference to a [discrete] point in time.'' \autocite{liao2017}
Thus, technology that underpins a payment system that achieves final settlement must be able to reflect this point in time on a technical level.
Traditional, centralized database systems are, without a doubt, able to do so:
After submission to such a system, from a discrete point in time onward, a transaction is recorded -- in fact, permanently so, as it is guaranteed that the system will never autonomously roll back said record.
Thus, traditional, centralized database systems can be said to provide what, in the remainder of this thesis, is referred to as ``technical finality''.

However, the same is not true for blockchain systems, which cannot provide this novel notion of technical finality -- at least so I argue.
Since the latter is implicitly assumed in the SFD, though, I argue that blockchain-based payment systems are incompatible with it and, hence, neither provide finality in a technical nor in a legal sense. After all, what good does it do to insulate a payment system's transactions from legal effects that may cause their reversal, if the very database that records such transactions may roll them back?
In other words, who benefits from a payment that is de-jure final but, de-facto, is not recorded, i.e., does not exist?

It can be summarized that the SFD regulates payment systems in a direct fashion, as it only applies to those that conform to its payment system definition, which cannot be considered in isolation from the underlying technology -- at least not when discussing blockchain technology.
Expectedly, as blockchain-based systems cannot be regulated directly, payment systems based on blockchain technology do not fall under the auspices of the SFD and, hence, cannot guarantee legal finality.

More importantly, though, it is argued that such systems would be unable to achieve legal finality even if the SFD were to formally apply to them, because the concept of finality that the SFD is based on implicitly assumes that the technology underpinning payment systems is able to provide final settlement in a technical sense, i.e., as a discrete point in time after which a transaction is permanently recorded.
I argue blockchain technology is unable to do so -- a claim I will justify in the following -- and, therefore, state that blockchain-based payment systems can neither guarantee technical nor legal finality.
