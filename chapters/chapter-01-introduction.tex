More than a decade after blockchain technology was originally introduced with Bitcoin, its applications go well beyond that of cryptocurrencies.
Most notably, general-purpose blockchain systems, e.g., Ethereum, allow for the definition of smart contracts, which, in turn, enable the issuance of alternative tokens next to a blockchain system's native cryptocurrency token.
Such tokens are used to digitally represent a plethora of assets, e.g., shares in a company or property rights.

At the same time, the idea of central bank digital currencies (CBDC), i.e., digital representations of legal tender issued by a central bank, is increasingly discussed in academia and in the banking community, with many central banks having published working papers or even prototypes.
More often than not, CBDCs are envisioned to be ``on-chain,'' i.e., in the form of blockchain tokens.

However, digital money based on blockchain technology implies that payment systems, including wholesale payment systems, are blockchain-based as well.
This has intensified the debate on whether blockchain technology is well-suited to underpin such systems.

While many facets of blockchain systems have been analyzed with regards to this, little work has been published on whether and how blockchain-based payment systems achieve final settlement.
However, settlement finality is a crucial property of payment systems, as it is directly related to the settlement risks that participants incur and even to systemic risk.

This thesis aims to close this knowledge gap.
To this end, the following three crucial questions are addressed:

\begin{description}
	\item[How is finality defined in the literature?]
		The first part of this thesis surveys the literature to derive a clear definition of finality, as the term has come to mean different things to different people.
	\item[Can blockchain systems provide legal finality?]
		As finality is traditionally defined in a legal sense, the second part of this thesis examines whether blockchain-based payment systems can provide said legal finality.
	\item[Can blockchain systems provide technical finality?]
		As there also is a technical aspect to finality that is traditionally assumed to ``just work,'' the third part of this thesis examines whether blockchain-based payment systems can achieve this novel notion of technical finality.
\end{description}

Based on the insights resulting from addressing these questions, a comprehensive account of finality in the context of blockchain-based payment systems is given.
